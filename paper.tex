\documentclass{article}

\usepackage{amsmath}
\usepackage{graphicx}
\usepackage{algorithmicx}
\usepackage{algpseudocode}

\title{Rubik's Cube Solvability Classes}
\date{March, 2020}
\author{Tulba-Lecu Theodor-Gabriel\\
    Dragomir Ioan\\
    email: [REDACTED] \\
    Institute: [REDACTED]}

\begin{document}
    
    \maketitle

    \pagenumbering{arabic}

    \begin{abstract}
        blah blah cube blah blah maths blah blah stupid.
    \end{abstract}

    \tableofcontents
    
    \newpage

    \section{Introduction}
    
        \subsection{History}
            
        
        \subsection{Notation}
            blah blah F F' and so on
        
        \subsection{Motivation}
            The motivation for this research came when the we were playing 
            around with a Rubik's Cube. At some point, one edge piece flew 
            away, and we put it back in the wrong orientation.
            Nevertheless we continued to solve the cube and observed that it
            was impossible to solve the cube. \\ \\
            We then asked ourselves the following question: 
            \textit{If we define a transition from one cube state to another,
            using only valid moves, an equivalence realtion, how many
            equivalence classes are there?} \\
            We quickly amde some assumptions and worte some code to bruteforce 
            all the states *add refference here*. We saw quite fast that the
            answer was 12, a result which corresponded both with our intuition 
            and with other results *add refferences*. But where did this 
            number come from? how does the answer change if we change the 
            size of the cube? \\ \\ 
            The purpose of this paper is to answer those questions.
    
    \section{Solving the 3x3x3 Rubik's}
        blah blah summary of the aproach
    
        \subsection{Reduction to last-layer}
            blah blah proof by contradiction and deterministic solution
        
        \subsection{Naive full search space exploration}

            Our first approach consisted in exploring all possible last layer
            states and modelling the possible transitions in between these. In
            order to simplify the search, we abstracted away the notion of
            single turn moves, thus only focusing on PLL and OLL algorithms
            given their documented existence, but not needing to simulate their
            constituent moves.
            For instance, in the case of the H permutation, it is superfluous to
            simulate the move sequence [M2 U M2 U2 M2 U M2] and keep track of
            all 26 cubies, given that we know the final result will be that the
            first 2 layers remain unchanged and the last layer edge cubies keep
            their orientation but swap positions with their opposite.
            Thus, it is sufficient to only encode these position and orientation
            transforms last layer algorithms.

            Through experimentation, we found that only a relatively minimal set
            of algorithms need to be implemented for full search space coverage.
            For example, all OLL algorithms can be reduced to a sequence of U
            turns and [F R U R' U' F'] sequences, and all PLL algorithms can
            be reduced to repeated U turns and Aa, Z, and Ua permutations.

            States are encoded by 4 edge and 4 corner states. Edges and corners
            are numbered in counter clockwise order starting with the right
            edge and back-right corner, as illustrated in the following figure.

            \begin{table}[h]
            \centering
            \begin{tabular}{|l|l|l|}
            \hline
            corner 1 & edge 1 & corner 0\\
            \hline
            edge 2 & & edge 0\\
            \hline
            corner 2 & edge 3 & corner 3\\
            \hline
            \end{tabular}
            \end{table}

            As a convention, we also number sticker colors according to what
            what edge they are on in the solved state. Color 0 is the color of
            the right face, color 1 is the color of the back face, etc.

            \begin{align}
            &\nonumber edgeStates[i] = (color, flipped) \\
            &\nonumber cornerStates[j] = (color, rotation) \\
            &\nonumber i, j \in \{0, 1, 2, 3\},\ color \in \{0, 1, 2, 3\} \\
            &\nonumber flipped = \begin{cases}
                0, & \text{if the yellow sticker is facing up}\\
                1, & \text{if the yellow sticker is facing outwards}
            \end{cases} \\
            &\nonumber rotation = \begin{cases}
                0, & \text{if the yellow sticker is facing up}\\
                1,\ 2, & \text{by successive counter-clockwise rotations}
            \end{cases}
            \end{align}
            
            We define two operators on these states: flipEdge(edge)
            which flips an edge's orientation, and rotateCorner(corner,
            third\_turns) which adds a certain number of third turns to a
            corner's rotation (addition modulo 3).
            By combining these two operators and permutations on the edges' and
            corners' positions we can achieve any last layer algorithm. For
            example, the Sune algorithm can be decomposed into the edge
            permutation,
            \begin{align}
            \begin{pmatrix}
            &edgeStates[0] &edgeStates[1] &edgeStates[2] &edgeStates[3] \\
            &edgeStates[2] &edgeStates[1] &edgeStates[3] &edgeStates[0]
            \end{pmatrix} \nonumber
            \end{align}
            three corner rotations,\\\\
            rotateCorner(cornerStates[1], 2)\\
            rotateCorner(cornerStates[2], 2)\\
            rotateCorner(cornerStates[3], 2)\\\\
            and two corner swaps, represented by the permutation:
            \begin{align}
            \begin{pmatrix}
            &cornerStates[0] &cornerStates[1] &cornerStates[2] &cornerStates[3] \\
            &cornerStates[2] &cornerStates[3] &cornerStates[0] &cornerStates[1]
            \end{pmatrix} \nonumber
            \end{align}

            To explore all possible states, we needed an efficient way to check
            whether a certain state was visited in the past. For this we created
            a simplistic hashing function which transforms a given last layer
            state to an integer, accounting for potential rotations of similar
            states. The hashing algorithm first rotates the cube so the 0 edge
            cubie is in the 0 position, and then employs a simple arithmetic
            coding (I might be way off and this might not be arithmetic coding
            at all) which multiplies an accumulator by 8 for edges and 12 for
            corners, adding the corresponding color and flip/rotation after each
            multiplication.

            \begin{algorithmic}
            \Repeat
                \State RotateClockwise()
            \Until {$edgeStates[0].color = 0$}
            \State $hash \gets edgeStates[0].flipped$
            \For {$i \gets 1, 3$}
                \State $hash \gets hash * 8 + edgeStates[i].flipped * 4 + edgeStates[i].color$
            \EndFor
            \For {$j \gets 0, 3$}
                \State $hash \gets hash * 12 + cornerStates[j].rotation * 4 + cornerStates[j].color$
            \EndFor \\
            \Return hash
            \end{algorithmic}

            Now that we can quickly identify identical positions, we use an
            associative array whose keys are state hashes and whose values are
            integers so that two states have the same value iff it is possible
            to reach one state starting at the other, and vice versa.

            We define a solvability class to be a maximal subset of the state
            set such that any two of its elements have a sequence of algorithms
            which take one state to the other. (this will probably be moved to
            some earlier section)

            Thus, this array's values can be used as identifiers for the
            discovered solvability classes.

            The search space exploration algorithm is as follows.

            ...

        
        \subsection{Giving an answer}
            \subsubsection{Corner orientation}
                3

            \subsubsection{Edge orientation}
                2

            \subsubsection{Edge position}
                2

            \subsubsection{Final independence}
                3*2*2=12
    
    \section{Solving 4x4x4}
        blah blah analyse differences from 3x3x3

        \subsection{Generalizing theorems}
            easy to say, hard to do

        \subsection{Solving the symmetry problem}
            blah blah parity blah blah symmetric slices from the center
        
        \subsection{Solve the geometric invariant}
            blah blah probably said something dumb here

        \subsection{Results for 4x4x4}
            Hmmm...

    \section{Solving NxNxN}
        blah blah why not take it further

        \subsection{Prove independece of non-symmetric layers}
            probably true

        \subsection{Reduce to 3x3x3 and 4x4x4 cases}
            not to hard, maybe induction

        \subsection{Flex with vector spaces to calculate the answer}
            lol, mathy boiiiis
    
    \section{Generalizing to cuboids}
        blah blah even crazier idea

    \section{Solve the problem for other platonic solids}
        lol what is this even.

\end{document}
