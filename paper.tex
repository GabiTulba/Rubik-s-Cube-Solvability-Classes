\documentclass{article}

\usepackage{amsmath}
\usepackage{graphicx}

\title{Rubik's Cube Solvability Classes}
\date{March, 2020}
\author{Tulba-Lecu Theodor-Gabriel\\
	Dragomir Ioan\\
	email: [REDACTED] \\
	Institute: [REDACTED]}

\begin{document}
    
    \maketitle

    \pagenumbering{arabic}

    \begin{abstract}
        blah blah cube blah blah maths blah blah stupid.
    \end{abstract}

    \tableofcontents
    
    \newpage

    \section{Introduction}
    
        \subsection{History}
            blah blah 1980's blah blah Rubik blah blah Hungary
        
        \subsection{Notation}
            blah blah F F' and so on
        
        \subsection{Question}
            blah blah ECSC story here
    
    \section{Solving the 3x3x3 Rubik's}
        blah blah summary of the aproach
    
        \subsection{Reduction to last-layer}
            blah blah proof by contradiction and deterministic solution
        
        \subsection{Searchspace bruteforce result}
            blah blah only code here
        
        \subsection{Giving an answer}
            \subsubsection{Corner orientation}
                3

            \subsubsection{Edge orientation}
                2

            \subsubsection{Edge position}
                2

            \subsubsection{Final independence}
                3*2*2=12
    
    \section{Solving 4x4x4}
        blah blah analyse differences from 3x3x3

        \subsection{Generalizing theorems}
            easy to say, hard to do

        \subsection{Solving the symmetry problem}
            blah blah parity blah blah symmetric slices from the center
        
        \subsection{Solve the geometric invariant}
            blah blah probably said something dumb here

        \subsection{Results for 4x4x4}
            Hmmm...

    \section{Solving NxNxN}
        blah blah why not take it further

        \subsection{Prove independece of non-symmetric layers}
            probably true

        \subsection{Reduce to 3x3x3 and 4x4x4 cases}
            not to hard, maybe induction

        \subsection{Flex with vector spaces to calculate the answer}
            lol, mathy boiiiis
    
    \section{Generalizing to cuboids}
        blah blah even crazier idea

    \section{Solve the problem for other platonic solids}
        lol what is this even.

\end{document}
