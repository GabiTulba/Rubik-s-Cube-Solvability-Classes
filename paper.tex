\documentclass{article}

\usepackage{amsmath}
\usepackage{graphicx}

\title{Rubik's Cube Solvability Classes}
\date{March, 2020}
\author{Tulba-Lecu Theodor-Gabriel\\
	Dragomir Ioan\\
	email: [REDACTED] \\
	Institute: [REDACTED]}

\begin{document}
    
    \maketitle

    \pagenumbering{arabic}

    \begin{abstract}
        blah blah cube blah blah maths blah blah stupid.
    \end{abstract}

    \tableofcontents
    
    \newpage

    \section{Introduction}
    
        \subsection{History}
            
        
        \subsection{Notation}
            blah blah F F' and so on
        
        \subsection{Motivation}
            The motivation for this research came when the we were playing 
            around with a Rubik's Cube. At some point, one edge piece flew 
            away, and we put it back in the wrong orientation.
            Nevertheless we continued to solve the cube and observed that it
            was impossible to solve the cube. \\ \\
            We then asked ourselves the following question: 
            \textit{If we define a transition from one cube state to another,
            using only valid moves, an equivalence realtion, how many
            equivalence classes are there?} \\
            We quickly amde some assumptions and worte some code to bruteforce 
            all the states *add refference here*. We saw quite fast that the
            answer was 12, a result which corresponded both with our intuition 
            and with other results *add refferences*. But where did this 
            number come from? how does the answer change if we change the 
            size of the cube? \\ \\ 
            The purpose of this paper is to answer those questions.
    
    \section{Solving the 3x3x3 Rubik's}
        blah blah summary of the aproach
    
        \subsection{Reduction to last-layer}
            blah blah proof by contradiction and deterministic solution
        
        \subsection{Searchspace bruteforce result}
            blah blah only code here
        
        \subsection{Giving an answer}
            \subsubsection{Corner orientation}
                3

            \subsubsection{Edge orientation}
                2

            \subsubsection{Edge position}
                2

            \subsubsection{Final independence}
                3*2*2=12
    
    \section{Solving 4x4x4}
        blah blah analyse differences from 3x3x3

        \subsection{Generalizing theorems}
            easy to say, hard to do

        \subsection{Solving the symmetry problem}
            blah blah parity blah blah symmetric slices from the center
        
        \subsection{Solve the geometric invariant}
            blah blah probably said something dumb here

        \subsection{Results for 4x4x4}
            Hmmm...

    \section{Solving NxNxN}
        blah blah why not take it further

        \subsection{Prove independece of non-symmetric layers}
            probably true

        \subsection{Reduce to 3x3x3 and 4x4x4 cases}
            not to hard, maybe induction

        \subsection{Flex with vector spaces to calculate the answer}
            lol, mathy boiiiis
    
    \section{Generalizing to cuboids}
        blah blah even crazier idea

    \section{Solve the problem for other platonic solids}
        lol what is this even.

\end{document}
